
\documentclass[a4paper,11pt]{article}
\usepackage[T1]{fontenc}
\usepackage[pdftitle={IF XML metadata format},
            colorlinks=true,
linkcolor=blue,
            bookmarks=true,
bookmarksopen=true,
bookmarksnumbered=false,
pdfhighlight=/P,
pdfauthor={Andrew Hunter},
pdfpagemode={UseOutlines},
pdfstartview=FitH]{hyperref}
\usepackage{times}
\usepackage{verbatim}
\usepackage{bleh}

\newfont{\sst}{cmssbx10 at 24pt}

\setlength{\textwidth}{16cm}
\setlength{\oddsidemargin}{0.5cm}
\setlength{\evensidemargin}{0cm}

\setlength{\topmargin}{-1cm}
\setlength{\headheight}{12pt}
\setlength{\headsep}{6pt}
\setlength{\textheight}{25cm}

\setlength{\parindent}{0cm}
\setlength{\parskip}{1.4ex plus0.5ex minus0.3ex}

\begin{document}

\begin{titlepage}

\begin{flushright}
**Draft**
\end{flushright}

\rule{0cm}{6cm}

\begin{flushright}
\sst
Interactive Fiction XML metadata format\rm\\
\rule{16cm}{1.2mm}\\
Version 1.0 (Draft)\\
\today\\
by Andrew Hunter
\end{flushright}
\vfill

\end{titlepage}

\tableofcontents

\section{Introduction}

The original inspiration of this format is an improvement on the existing interactive fiction
cataloguing system in Zoom. The original design for this used a text file with an arbitrary
format, mainly designed for storing interpreter settings. It was also capable of storing
some limited metadata along with stories, in particular a full title for the story, which
was used in the interpreter's title bar, and also to provide a simple menu of stories.

This format is intended to provide a more advanced indexing system for stories. It
deals specifically with metadata aspects of stories, leaving interpreter configuration
issues up to specific interpreters. This release (1.0) will be implemented in the 1.0.2
release of Zoom. The initial emphasis is on Z-Code stories, but the intention is to allow
the indexing of stories in any of the formats currently in use.

A simpler format (0.9) was originally used by Zoom. This new version is based on the
results of a discussion on the Inform-developers mailing list.

\subsection{Who should use this format?}

Initially, this format will be used by Zoom to provide a searchable list of stories. However,
I hope eventually to encourage archive maintainers and other interpreter authors to make
use of the format. In addition, story authors may want to provide an XML description file
along with their stories.

\section{The format}

All metadata index files are in XML format, and so begin with a <?xml version="1.0"?>
line. The namespace URI is http://www.logicalshift.org.uk/IF/metadata. The metadata may
be extended by additional modules, which must have their own unique namespace. Parsers
must always identify namespaces by URI and not by prefix.

The metadata for a story may come from many different files. For example, an archive
site may provide an index of games, a review site may provide reviews, the game author
may provide 'feelies' (resources associated with the story), and the user may
provide notes and comments. Modules are intended to be 'functionally unique', providing
the data for one of these specific domains. Therefore, when an application fetches metadata
from multiple locations, it might use the entire resources module from one place and the
review module from another, but won't merge parts of the resources section from one site
with those provided by another.

\subsection{The root element: ifindex}

This tag specifies the version of the IF metadata format with the 'version' attribute (0.9 for a
file conforming to this document). Usually, this should also specify the namespace URI
and XML schema location.

\begin{example}
  \begin{verbatim}
<ifindex version="1.0"
  xmlns="http://www.logicalshift.org.uk/IF/metadata">
  \end{verbatim}
\end{example}

The <ifindex> tag contains <story> tags, one for each story that is defined in the file.

\subsection{The story tag: story}

The <story> tag is used to specify the data associated with a particular story. It must contain
at least one <identification> section, and may contain one or more metadata tags. It appears
only as a child of the ifindex tag.

The <identification> section may be omitted if the metadata is associated with a game in
some other fashion (for example if it is contained in a Blorb file).

\subsection{The story identification tag: identification}

The <identification> tag appears as part of a <story> section, and must contain a <format>
tag identifying the format of the story, and some data providing a unique ID for the story.
It appears only as a child of <story>, which must contain at least one of these tags, but
may contain many if a story has several different versions. The <id> tag is a synonym for
the <identification> tag.

The <format> tag contains textual data identifying the format of the story. This version
of the specification defines identification tags for the following formats:

\begin{description}
\item[zcode] A Z-Code story of the type released by Infocom in the 1980s, or, more
recently, compiled with the Z-Code version of Inform.
\item[glulx] A glulx story of the type compiled by the Glulx version of Inform.
\end{description}

The following additional types are reserved for other existing formats, and may have
identification tags defined in the future:

\begin{description}
\item[tads] A TADS story compiled by the TADS compiler.
\item[hugo] A HUGO story compiled by the HUGO compiler.
\item[alan] An Alan story
\item[adrift] An Adrift story
\item[level9] A story in the Level 9 format
\item[AGT] A story in the AGT format
\item[magscrolls] A story in the magnetic scrolls format
\item[advsys] A story in the advsys format
\end{description}

\subsubsection{Example}

\begin{example}
\begin{verbatim}
<story>
  <identification>
    <format>zcode</format>
    <zcode>
      <serial>871125</serial>
      <release>52</release>
      <checksum>4b37</checksum>
    </zcode>
    <md5>d82484e664a8c328364e9814d1f73c60</md5>
  </identification>
</story>
\end{verbatim}
\end{example}

\subsubsection{Z-Code identification tag: zcode}

The <zcode> tag is used only when the <format> value is 'zcode'. In that case, it must
appear exactly once as a child of  an <indentification> tag. It should contain the following
elements:

\begin{description}
\item[<serial>] The textual serial number of the story. This is usually a six-digit string
specifying the date of release, such as 871125.
\item[<release>] The release number of the story (for example, 52).
\item[<checksum>] (Optional, but recommended) The checksum of the story, in 
hexadecimal format, as specified in the header of the story (for example, 4b37). This field is 
optional, but note that the serial/release numbers are not guaranteed to be unique, especially 
with more recent stories.
\end{description}

\subsubsection{Glulx identification}

ULX format files do not contain any specific story identification information. However,
those compiled by Inform may contain a short 'Info' section located after the header. The
<glulx> section is therefore optional, and only applies to stories compiled with Inform.
An <md5> section should be used to more precisely identify a Glulx story. The <glulx>
section may contain the following tags:

\begin{description}
\item[<serial>] The textual serial number of the story. This is usually a six-digit string
specifying the date of release, such as 030501.
\item[<release>] The release number of the story (for example, 2).
\end{description}

\subsubsection{MD5 identification}

Any story format may have an MD5 section instead of a format-specific identification. 
This is a hexadecimal representation of the
MD5 checksum of a story file. In the case of a Blorb file, this is the MD5 checksum of
the executable portion of the file (the Z-Code or Glulx code file that it contains) rather
than the entire file. There are several utilities available for generating MD5 checksums,
for example the 'md5' utility that comes with OpenSSL.

A story format that has a format-specific means of identification will probably want to
provide both an MD5 and the format-specific ID tags.

\subsubsection{Resource identification}

A 'story' might be considered as spread across several files. Typically, IF systems that do
this distribute files into a 'game' file containing the actual story data, and some kind of
'resources' file containing additional information such as images and sounds. With modern
graphical operating systems, it is desirable that a game resource can be associated with the
game that created it. To enable this, this specification provides several extra identification
formats that identify resources. The contents of the resources module can be used to find
the actual resources associated with a game.

\subsection{The metadata tags}

Each metadata tag may appear zero or one times as a child of  the <story> tag. The following
tags are defined:

\begin{description}
\item[<title>] The title of the story (for example, Zork 1, Solid Gold Edition)
\item[<author>] The author of the story
\item[<genre>] The genre of the story (Fantasy)
\item[<published>] The date the story was first published in YYYYMMDD format (19810505)
\item[<revision>] The revision number of the game
\item[<headline>] The headline of the story (Infocom interactive fiction - a fantasy story)
\item[<group>] The 'group' the story is in (for example, 'Infocom', 'Competition 1998', etc)
\item[<zarfian>] The story's rating according to the 'Zarfian' scale ('Merciful', 'Polite', 
'Tough', 'Nasty', 'Cruel')
\item[<teaser>] The story's teaser text
\item[<series>] Specifies the series that a game is part of
\item[<seriesnumber>] Specifies the number in the series the game is
\item[<serieslength>] Specifies the length of the game series
\end{description}

The 'Zarfian' rating system is described by Andrew Plotkin as follows:

\begin{description}
\item[Merciful] cannot get stuck
\item[Polite] can get stuck or die, but it's immediately obvious that you're stuck or dead
\item[Tough] can get stuck, but it's immediately obvious that you're about to do something 
irrevocable
\item[Nasty] can get stuck, but when you do something irrevocable, it's clear 
\item[Cruel] can get stuck by doing something which isn't obviously irrevocable (even after the act)
\end{description}

Comments and teasers may contain newlines, specified by a <br /> tag: all other tags are
single line values. No tag may appear more than once. The genre can be any textual value,
but Zoom suggests the following list of 'standard' genres:

\begin{itemize}
\item Fantasy
\item Science fiction
\item Horror
\item Fairy tale
\item Surreal
\item Mystery
\item Romance
\item Historical
\item Humour
\item Parody
\item Speed-IF
\item Arcade
\item Interpreter abuse
\end{itemize}

\section{Optional modules}

\subsection{The feelies module}

This module can be used to specify 'feelies': additional media associated with a game. The namespace URI for this module is 
http://www.logicalshift.org.uk/IF/metadata/feelies

The tags supported by this module are:

\begin{description}
\item[<icon>] A miniature picture associated with the game. This should be a base 64 encoded 
PNG file of 128x128 size, and can be used by an interpreter to display an image associated with
a game file.
\item[<image>] A URL of an image associated with the game. Images should be of PNG or
JPEG format.
\item[<pdf>] A URL of a PDF document associated with the game. Can have an optional 'page' attribute to indicate that the
pdf file should be opened to a particular page (intended to be used with the masterpieces
CD documentation).
\end{description}

All of these tags should have a 'title' attribute, indicating a name that should be displayed
in association with the feelie.

\subsection{The comments module}

This module is intended for data designed to be set by the user of a game, and should not appear in metadata files generated by
game or interpreter authors. Its namespace URI is http://www.logicalshift.org.uk/IF/metadata/comments

The tags supported by this module are:

\begin{description}
\item[<comment>] Any user comments on the story
\item[<rating>] The user rating, a value between 0 and 10 where 0=worst and 10=best.
\item[<feelies>] External files, images and data associated with the game
\end{description}

\subsection{The resources module}

Certain game formats allow for external resource files, and nearly all of them specify
an external save game format. This module is used to describe the location of these files.

The namespace URI for this module is http://www.logicalshift.org.uk/IF/metadata/resources

The tags supported by this module are:

\begin{description}
\item[<story>] The URL of a file containing the story itself
\item[<graphics>] The URL of a file containing graphics for the story (in the case of a format like blorb, may be identical to <sounds>)
\item[<sounds>] The URL of a file containing sounds for the story (in the case of a format like blorb, may be identical to <graphics>)
\item[<savegame tag="Something" date="savedate">] The URL of a savegame for the story 
(multiple savegames are permitted.). The tag is a tag associated with the savegame
(Zoom uses a summary of the status bar contents here), and the date specifies when the
savegame was created.
\end{description}

Resources may be provided in an archive file, but are typically overridden by an interpreter
to point to local versions of the resources (particularly savegames). Tags may appear multiple
times if a story has many resources of a particular type.

\subsection{The review module}

Review sites may wish to provide review metadata. Reviews can be provided as plain text
or as an embedded XHTML document. Implementors might want to consider providing
an interface to switch between reviews when multiple sources provide them. 

The namespace URI for this module is http://www.logicalshift.org.uk/IF/metadata/review

The review tags are as follows:

\begin{description}
\item[<url>] A URL of a web page associated with this review.
\item[<author>] The name of the author(s) of the review
\item[<rating>] (Optional) A decimal value from 0-10 indicating the overall review rating
\item[<text>] The text of the review
\item[<xhtml>] (Optional replacement for text) The review as an XHTML format document
\end{description}

\section{Parsing notes}

Zoom provides an implementation of this standard (actually, at present, the 0.9 standard) in
the ifmetadata.c file provided as part of the source code. This can easily be used in other
projects: it depends on the expat library but nothing else. In addition to parsing the metadata,
it handles managing the results: adding new stories, saving the results, etc.

(MORE TO DO)

\section{Attaching metadata}

\subsection{To z-code games}

This section describes a technique by which XML metadata can be made part of a Z-Code
game. Other file formats may already have methods for attaching metadata. This technique
is similar to the ID3 tag method used with mp3 files.

Z-Code files can be any length: file size limits are only imposed by the range of locations
that can be addressed. As metadata does not need to be addressed by the game, it can be
appended to a story file, even if that file is at the upper limit of the size permitted by the
version (this is particularly important when large resources such as icons are provided).
Some interpreters may not support this with large files however, and game authors should
be made aware of this possibility (most I have inspected do, however).

Metadata should be the last thing in a file. However, this format specifies lead-in and 
lead-out so that it can appear elsewhere if necessary. It's expected that most implementations
will only appear at the end. Metadata is preceded by a 12 byte lead-in, consisting of
the 8-byte sequence 'IFMd0100' indicating metadata and the version number, followed
by a 4 byte big-endian length. The XML metadata itself follows this, followed by a 12
byte lead-out: the 4 byte big-endian length, followed by the sequence '0100IFMd'.
Implementations can look for the '0100IFMd' sequence at the end of a file to determine
if metadata is present.

(TODO: metadata and the story checksum?)

\subsection{In filesystem metadata}

\subsection{To archive sites}

\subsection{To review sites}

\end{document}
